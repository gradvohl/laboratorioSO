\chapter{Disparando vários processos}
Neste capítulo, veremos como criar processos a partir de outros processos utilizando as primitivas do sistema operacional.

\section{Primitiva fork}
A primitiva \texttt{fork()} é utilizada para, a partir de um processo, criar outro processo com o mesmo código e conteúdo de memória do primeiro. Na verdade, a primitiva \texttt{fork()} faz uma cópia do processo pai em um processo filho, fazendo com que ambos continuem a sua execução do ponto imediatamente posterior à  primitiva \texttt{fork()}.

A primitiva \texttt{fork()} tem três saídas distintas:
\begin{itemize}
\setlength{\itemsep}{1pt}\setlength{\parskip}{0pt}  \setlength{\parsep}{0pt}
\item \texttt{-1} se houve problemas (nesse caso o filho não é criado);
\item \texttt{0}, para o processo filho;
\item identificador do filho, para o processo pai.
\end{itemize}

Observe o programa a seguir e tente entender o funcionamento da primitiva \texttt{fork()}.

\lstinputlisting[style=MyCStyle]{./Programas/Processos/PaiFilho.c}

\subsection{Exercício}
Neste exercício, compilaremos o programa anterior e o executaremos.

Antes de compilar o programa, verifique se você está no diretório \texttt{Processos} onde se encontra o arquivo  \texttt{PaiFilho.c}, com o seguinte comando:

\begin{lstlisting}[style=MyBashStyle]
ls PaiFilho.c
\end{lstlisting}

Para compilar o programa utilize o comando a seguir:
\begin{lstlisting}[style=MyBashStyle]
gcc PaiFilho.c -o PaiFilho
\end{lstlisting}

Depois, execute o programa \texttt{./PaiFilho} e observe que dois processos serão criados. O processo Pai executará as linhas 47 a 51. Ao mesmo tempo, o processo filho executará as linhas 38 a 43.

\section{Criando processos zumbis}
Um processo zumbi é o processo que já terminou sua execução, mas continua na tabela de processos por algum motivo. Um desses motivos é que, por algum \textit{bug} no sistema operacional, a tabela de processos ainda não foi atualizada, eliminando o identificador do processo.

A princípio, um processo zumbi não é um problema sério para o sistema operacional. No entanto, a presença de zumbis pode indicar \textit{bugs} no sistema ou problemas de segurança do tipo \textit{Denial of Service}. Para ilustrar, no exemplo a seguir, forçaremos a criação de processos zumbis.

\lstinputlisting[style=MyCStyle]{./Programas/Processos/zumbi.c}

\subsection{Exercício}
Antes de compilar o programa \texttt{zumbi.c}, abra outra janela do terminal. Você precisará executar o comando \Comando{top} na segunda janela, enquanto o programa é executado na primeira.

Compile o programa \texttt{zumbi.c} com o seguinte comando:

\begin{lstlisting}[style=MyBashStyle]
gcc zumbi.c -o zumbi
\end{lstlisting}

Agora, execute o programa \texttt{./zumbi} em uma janela e na outra execute o comando a seguir:

\begin{lstlisting}[style=MyBashStyle]
top -p <id_pai> -p <id_filho>
\end{lstlisting}

Os valores para \texttt{<id\_pai>} e \texttt{<id\_filho>} serão fornecidos pelo programa \texttt{zumbi} na primeira janela.

\section{Processos pai e filho diferentes}
A princípio, a primitiva \texttt{fork()} cria um processo filho exatamente igual ao seu processo pai. Entretanto, cada um deles fica em um espaço de endereços de memória diferente.

Contudo, há situações em que é necessário que cada processo -- pai e filho -- execute códigos diferentes. No exemplo a seguir, ilustra-se a primitiva \texttt{execvp()} para executar programas diferentes a partir de um determinado processo. 

A primitiva \texttt{execvp()} faz parte de uma família de primitivas que substitui a imagem do processo atual por uma nova. A imagem de um processo são os códigos (programa) que aquele processo executa e os respectivos dados.

A sintaxe da primitiva \texttt{execvp()} é a seguinte:

\begin{lstlisting}[style=MyCStyle, frame=none, numbers=none]
int execvp(const char *file, char *const argv[]);
\end{lstlisting}
 
onde: 
\begin{itemize}
    \item O valor de retorno é sempre \texttt{-1}. Mas, se isso acontecer, significa que houve um erro na execução da primitiva;
    \item \texttt{file} é nome do programa;
    \item \texttt{argv[]} é um vetor de \textit{strings} com os argumentos do programa. Importante: a primeira posição do vetor \texttt{argv} deve ter o caminho completo para o programa e última posição do vetor deve ter valor \texttt{NULL}.
 \end{itemize}
 
 O exemplo a seguir ilustra o programa que representa os processos pai e filho. 
 \newpage
 
 \section*{\texttt{pai.c}}
 \lstinputlisting[style=MyCStyle]{./Programas/Processos/pai.c}
 
 \section*{\texttt{filho.c}}
 \lstinputlisting[style=MyCStyle]{./Programas/Processos/filho.c}
 
 \subsection{Exercício}
 Compile os programas \texttt{pai.c} e \texttt{filho.c} separadamente com os seguintes comandos:

\begin{lstlisting}[style=MyBashStyle]
gcc pai.c -o pai
gcc filho.c -o filho
\end{lstlisting}


Agora execute o programa \texttt{./pai} e veja o resultado.