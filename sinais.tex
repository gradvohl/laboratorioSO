\chapter{Tratamento de Sinais}

Sinais são usados para notificar um processo ou segmento de um evento particular. Pode-se comparar o tratamento de sinais com interrupções de hardware, que ocorrem quando um subsistema de hardware, por exemplo uma interface de E/S de disco, gera uma interrupção para o processador quando a E/S é concluída.

Este evento, por sua vez, faz com que o processador chame um tratador de interrupções. Assim, o processamento subsequente pode ser feito no sistema operacional com base na fonte e da causa da interrupção.

Observe como isso pode ser feito no programa \texttt{sinais.c} a seguir

\lstinputlisting[style=MyCStyle]{./Programas/Sinais/sinais.c}

\section{Exercício}
 Antes de compilar o programa, mude para o diretório onde se encontra o arquivo \texttt{sinais.c}, com o seguinte comando:

\begin{lstlisting}[style=MyBashStyle]
cd ../Sinais
\end{lstlisting}

Agora para compilar o programa utilize o comando a seguir:
\begin{lstlisting}[style=MyBashStyle]
gcc sinais.c -o sinais.o 
\end{lstlisting}

Depois de compilado, será necessário abrir uma segunda janela do terminal. Na primeira janela, você executará o programa \texttt{./sinais.o}. 

Depois que o programa entrar em execução, tente pressionar as teclas \keys{\ctrl + c} para ver se o programa termina.

Para encerrar de fato o programa, na segunda janela, utilize o comando \Comando{kill} para enviar um sinal de término para o programa. Para isso, utilize o comando a seguir:

\begin{lstlisting}[style=MyBashStyle]
kill -QUIT <pid>
\end{lstlisting}

\noindent onde \texttt{<pid>} é o identificador do processo na primeira janela.

Importante: para saber o identificador do processo \texttt{./sinais} que está em execução na primeira janela, use o comando a seguir:

\begin{lstlisting}[style=MyBashStyle]
ps -ef | grep sinais.o
\end{lstlisting}
