\chapter{Tratamento de Sinais}

Sinais são usados para notificar um processo ou segmento de um evento particular. Pode-se comparar o tratamento de sinais com interrupções de hardware, que ocorrem quando um subsistema de hardware -- por exemplo, uma interface de entrada ou saída (E/S) de disco -- gera uma interrupção para o processador quando a E/S é concluída.

Esse evento, por sua vez, faz com que o processador chame um tratador de interrupções. Assim, o processamento subsequente pode ser feito no sistema operacional com base na fonte e da causa da interrupção.

\section{O comando \texttt{kill}}
Apesar do nome, no Linux, o usuário pode usar o comando \Comando{kill} para enviar sinais para um processo em execução. Para ver uma lista de sinais que podem ser enviados execute o comando a seguir.

\begin{lstlisting}[style=MyBashStyle]
kill -l
\end{lstlisting}

Dentre os vários sinais que podem ser enviados, aqueles que estão na Tabela~\ref{tab:sinais} se destacam. A coluna \textbf{Valor} indica os valores inteiros dos respectivos sinais; e na coluna \textit{Ações}, estão indicadas as ações padrão que acontecerão logo após a ocorrência do sinal. Essas ações podem ser \texttt{Term}, que indica que o processo deve terminar; \texttt{Ign}, que informa que o sinal deve ser ignorado; \texttt{Core} que aponta que o processo deve ser terminado e uma imagem da sua memória será armazenada em disco; \texttt{Stop} que indica que o processo será suspenso; e \texttt{Cont}, que informa que o processo deve continuar.

\newpage

\begin{table}[!htb]
\caption{Tabela de sinais}\label{tab:sinais}
\begin{tabular}{@{}lccp{11.3cm}@{}}
\toprule
\textbf{Sinal}   & \textbf{Valor(es)}                        & \textbf{Ação} & \textbf{Descrição}\\ \midrule
\multirow{2}{*}{\texttt{SIGHUP}}  & \multirow{2}{*}{1} & \multirow{2}{*}{\texttt{Term}} & \textit{Hangup} detectado no terminal de controle ou morte do processo de controle. \\
\texttt{SIGINT}  & 2 & \texttt{Term} & Interrupção do process, via teclado (\keys{\ctrl + c}). \\
\texttt{SIGQUIT} & 3 & \texttt{Core} & Saída (\textit{quit}) pelo teclado.\\
\texttt{SIGKILL} & 9 & \texttt{Term} & Sinal de \textit{kill}. \\
\texttt{SIGTERM} & 15 & \texttt{Term} & Sinal de término.\\
\texttt{SIGCHLD} & 20, 17, 18 & \texttt{Ign}  & Processo filho parou ou terminou. \\
\texttt{SIGSTOP} & 17, 19, 23 & \texttt{Stop} & Para o processo. \\
\texttt{SIGCONT} & 19, 18, 25 & \texttt{Cont} & Continua, se parou. \\
\texttt{SIGTSTP} & 18, 20, 24 & \texttt{Stop} & Para o processo pelo terminal.\\ \bottomrule
\end{tabular}
\end{table}

\subsection{Exercicio}
Neste primeiro exercício, usaremos um programa simples que apenas imprime pontos na tela. O código para esse programa é o seguinte.

\lstinputlisting[style=MyCStyle]{./Programas/Sinais/ponto.c}

Antes de compilar o programa, mude para o diretório onde se encontra o arquivo \texttt{sinais.c}, com o seguinte comando:

\begin{lstlisting}[style=MyBashStyle]
cd ../Sinais
\end{lstlisting}

Para compilar o programa utilize o comando a seguir:
\begin{lstlisting}[style=MyBashStyle]
gcc pontos.c -o pontos 
\end{lstlisting}

Após compilado, será necessário abrir uma segunda janela do terminal. Na primeira janela, você executará o programa \texttt{./pontos}. Certifique-se de que na segunda janela você está no diretório \texttt{Sinais} (para isso, use o comando \Comando{pwd}).

Agora, na segunda janela, utilize o comando a seguir para descobrir qual é o identificador do processo \textit{./pontos} que você instanciou na primeira janela. Note que o identificador é o número que está na segunda coluna da saída do comando.

\begin{lstlisting}[style=MyBashStyle]
ps -ef | grep pontos
\end{lstlisting}

Após descobrir o identificador, na segunda janela use o comando \Comando{kill} para parar o processo instanciado na primeira janela. Para isso, use o comando a seguir, substituindo \texttt{<pid>} pelo identificador do processo \texttt{pontos}.

\begin{lstlisting}[style=MyBashStyle]
kill -s STOP <pid>
\end{lstlisting}

Observe que, na primeira janela, o processo \texttt{pontos} está parado. Para que esse processo retorne, use o comando a seguir, substituindo \texttt{<pid>} pelo identificador do processo.

\begin{lstlisting}[style=MyBashStyle]
kill -s CONT <pid>
\end{lstlisting}

Para \enquote{matar} definitivamente o processo, use o comando a seguir.

\begin{lstlisting}[style=MyBashStyle]
kill -s KILL <pid>
\end{lstlisting}


\section{Interceptação de sinais via processo}
Utilizando comandos específicos, os processos podem interceptar alguns sinais enviados a eles a partir do sistema operacional. Isso pode ser útil quando o processo quer tratar esses sinais, antes que o sistema operacional os trate definitivamente.

No programa \texttt{sinais.c} a seguir, veremos como interceptar os sinais enviados a esse processo. Para isso, será criado um procedimento específico chamado \texttt{trataSinal} que será responsável pela interceptação e tratamento de alguns sinais.

\lstinputlisting[style=MyCStyle]{./Programas/Sinais/sinais.c}

\subsection{Exercício}
 
 Para esse segundo exercício, precisamos compilar o programa \texttt{sinais} com o comando a seguir:
 
\begin{lstlisting}[style=MyBashStyle]
gcc sinais.c -o sinais 
\end{lstlisting}

Após compilado, será necessário abrir uma segunda janela do terminal. Na primeira janela, você executará o programa \texttt{./sinais}. 

Quando o programa entrar em execução, tente pressionar as teclas \keys{\ctrl + c} para ver se o programa termina.

Para encerrar de fato o programa, na segunda janela, utilize o comando \Comando{kill} para enviar um sinal de término para o programa. Para isso, utilize o comando a seguir:

\begin{lstlisting}[style=MyBashStyle]
kill -QUIT <pid>
\end{lstlisting}

\noindent onde \texttt{<pid>} é o identificador do processo na primeira janela.

Importante: para saber o identificador do processo \texttt{./sinais} que está em execução na primeira janela, use o comando a seguir:

\begin{lstlisting}[style=MyBashStyle]
ps -ef | grep sinais
\end{lstlisting}
