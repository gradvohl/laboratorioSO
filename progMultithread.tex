\chapter{Programação \textit{Multithread}}
Uma outra forma de fazer duas ou mais tarefas ao mesmo tempo é utilizado \textit{multithreads}. Conforme já discutido em sala de aula, \textit{multithreads} são diferentes linhas de execução em um processo.

Existem algumas bibliotecas para trabalhar com \textit{multithreads}. Por exemplo, a POSIX \textit{Threads} (PThreads) -- que utilizaremos neste tutorial -- e a OpenMP, cujo paradigma é diferente da PThreads.

Neste tutorial, serão utilizadas as seguintes primitivas  da biblioteca PThreads:

\begin{itemize}
\setlength{\itemsep}{1pt}\setlength{\parskip}{0pt}  \setlength{\parsep}{0pt}
\item \texttt{pthread\_create()}: responsável pela criação de uma \textit{thread}.
\item \texttt{pthread\_exit()}: responsável por retornar um valor de uma \textit{thread}.
\item \texttt{pthread\_join()}: adiciona uma barreira para aguardar por uma segunda \textit{thread};
\item \texttt{pthread\_self()}: obtém o identificador da \textit{thread}.
\end{itemize}

Para ilustrar observe o programa \texttt{thrd.c} a seguir. Ele dispara duas \textit{threads} que ``dormem'' um tempo aleatório.

\section*{thrd.c}
\lstinputlisting[style=MyCStyle]{./Programas/Thread/thrd.c}

A definição das funções chamadas pelo programa principal estão no arquivo a seguir.

\section*{funcoes.c}
\lstinputlisting[style=MyCStyle]{./Programas/Thread/funcoes.c}


\section{Exercício}
Compile e execute o programa anterior. Antes de compilar o programa, mude para o diretório onde se encontram os arquivos \texttt{funcoes.c} e \texttt{thrd.c} , com o seguinte comando:

\begin{lstlisting}[style=MyBashStyle]
cd ../Thread
\end{lstlisting}

Para compilar, utilize a seguinte linha de comando:

\begin{lstlisting}[style=MyBashStyle]
gcc -lpthread funcoes.c thrd.c -o thrd
\end{lstlisting}

Observação: a chave \texttt{-lpthread} indica que será usada a biblioteca \texttt{pthread} para Linux.