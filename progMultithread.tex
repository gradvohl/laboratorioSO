\chapter{Programação \textit{Multithread}}
Outra forma de fazer com que duas ou mais tarefas sejam executadas \enquote{ao mesmo tempo} é utilizando \textit{multithreads}. Conforme já discutido em sala de aula, \textit{multithreads} são diferentes linhas de execução em um processo. Portanto, dois ou mais trechos de código de um mesmo processo podem estar em execução ao mesmo tempo.


\section{Biblioteca POSIX \textit{Threads}}
Existem algumas bibliotecas para trabalhar com \textit{multithreads}. Por exemplo, a POSIX \textit{Threads} (PThreads) -- que utilizaremos neste tutorial -- e a OpenMP, que adota um paradigma é diferente da PThreads.

Neste tutorial, serão utilizadas as seguintes primitivas  da biblioteca PThreads:

\begin{itemize}
\setlength{\itemsep}{1pt}\setlength{\parskip}{0pt}  \setlength{\parsep}{0pt}
\item \texttt{pthread\_create()}: responsável pela criação de uma \textit{thread}.
\item \texttt{pthread\_exit()}: responsável por retornar um valor de uma \textit{thread}.
\item \texttt{pthread\_join()}: adiciona uma barreira para aguardar por uma segunda \textit{thread}.
\item \texttt{pthread\_self()}: obtém o identificador da \textit{thread}.
\end{itemize}

Na biblioteca PThreads em particular, os \textit{threads} são implementados como funções, com uma \enquote{assinatura} específica. Essa \enquote{assinatura} é um padrão que as funções devem adotar na sua declaração, conforme ilustra a Figura~\ref{fig:assinaturaThread} a seguir. 

\begin{figure}[!htb]
    \centering
    \includegraphics[width=.8\textwidth]{AssinaturaThread.png}
    \caption{Assinatura padrão de um \textit{thread}.}
    \label{fig:assinaturaThread}
\end{figure}

Note que a função que implementa um \textit{thread} deve, obrigatoriamente, retornar o tipo \enquote{\mbox{\texttt{void *}}} e receber como parâmetro um tipo \enquote{\mbox{\texttt{void *}}}. Tanto o nome da função, quanto o nome da variável passada como parâmetro pode ser definidos pelo programador.

A utilização do tipo \enquote{\mbox{\texttt{void *}}} possibilita ao programador a utilização de um endereço para qualquer tipo de dado. Contudo, para evitar que o compilador lance advertências, é preciso fazer uma conversão de tipos (\textit{cast}) para usar o conteúdo do endereço utilizado.

\section{Sintaxe das principais primitivas na biblioteca PThreds}

Nesta seção, detalharemos as sintaxes das principais primitivas citadas na seção anterior. A primeira dessas primitivas é a \texttt{pthread\_create}, cuja sintaxe é a seguinte:

\begin{lstlisting}[style=MyCStyle, frame=none, numbers=none]
    int pthread_create(pthread_t *restrict thread,
                       const pthread_attr_t *restrict attr,
                       void *(*start_routine)(void *),
                       void *restrict arg);
\end{lstlisting}
 
onde: 
\begin{itemize}
    \item \texttt{thread} é o endereço da variável que armazenará o identificador do \textit{thread} criado.
    \item \texttt{attr} são os atributos do \texttt{thread}. Aqui, a recomendação é utilizar \texttt{NULL} para o \textit{thread} ser criado com os atributos padrão. 
    \item \texttt{start\_routine} é o nome da função que implementa o \textit{thread}.
    \item \texttt{arg} é o endereço da variável que armazena os argumentos (parâmetros) para o thread. Se não houver parâmetros a passar, utilize \texttt{NULL}.
 \end{itemize}

A chamada à primitiva \texttt{pthread\_create} retorna zero caso não haja erro ou um código de erro (maior que zero), caso contrário.

Por sua vez, a primitiva \texttt{pthread\_exit} é utilizada dentro da função que implementa o \textit{thread}. Essa primitiva é equivalente ao \texttt{return} no fim de uma função. A sintaxe é a seguinte:

\begin{lstlisting}[style=MyCStyle, frame=none, numbers=none]
       void pthread_exit(void *retval);
\end{lstlisting}

onde: 
\begin{itemize}
    \item \texttt{retval} contém o endereço da variável que armazena o valor ou valores de retorno do \textit{thread}.
\end{itemize}

Note que a variável que contém o valor ou valores de retorno precisa ser criada dinamicamente dentro do \textit{thread}. Afinal, variáveis locais deixam de existir quando a função encerra.

Por fim, a primitiva \texttt{pthread\_join} é fundamental para sincronizar os \textit{threads} secundários com o \textit{thread} principal. Essa primitiva é implementa um mecanismo de sincronização \enquote{barreira}. Isso significa que o \textit{thread} que chamou a primitiva \texttt{pthread\_join} ficará bloqueado até que o \textit{thread} cujo identificador é indicado nos parâmetros da primitiva termine. 

Aliás, essa é uma razão forte para usar a primitiva \texttt{pthread\_exit}, pois a utilização da primitiva \texttt{pthread\_exit} ao final de um \textit{thread} sinalizará, para outros \textit{threads} que aquele \textit{thread} terminou. A sintaxe da primitiva \texttt{pthread\_join} é a seguinte:

\begin{lstlisting}[style=MyCStyle, frame=none, numbers=none]
       int pthread_join(pthread_t thread, void **retval);
\end{lstlisting}

onde: 
\begin{itemize}
    \item \texttt{thread} é a variável que armazena o identificador do \textit{thread} cuja finalização se aguarda.
    \item \texttt{retval} contém o endereço da variável que armazenará o valor ou valores de retorno do \textit{thread} cuja finalização se aguarda.
\end{itemize}

Note que a variável \texttt{retval} é do tipo \enquote{\texttt{void **}}. Isso significa que \texttt{retval} deve ser um ponteiro e, mesmo assim, é preciso passar o endereço desse ponteiro (com o caractere `\texttt{\&}').


\section{Exemplo simples de utilização da biblioteca PThreads}
Para ilustrar a utilização da biblioteca PThreads, começaremos com um exemplo muito simples. Observe o programa \texttt{thrd.c} a seguir. O programa dispara duas \textit{threads} que \enquote{dormem} um tempo aleatório. 

Atente para os comentários que aparecem no código. Esses comentários explicam a utilização das primitivas da biblioteca PThreads.

\section*{thrd.c}
\lstinputlisting[style=MyCStyle]{./Programas/Thread/thrd.c}

A definição das funções chamadas pelo programa principal estão no arquivo a seguir.

\section*{funcoes.c}
\lstinputlisting[style=MyCStyle]{./Programas/Thread/funcoes.c}



\subsection{Exercício}
Compile e execute o programa anterior. Antes de compilar o programa, mude para o diretório onde se encontram os arquivos \texttt{funcoes.c} e \texttt{thrd.c} , com o seguinte comando:

\begin{lstlisting}[style=MyBashStyle]
cd ../Thread
\end{lstlisting}

Para compilar e executar, utilize as seguintes linhas de comando:

\begin{lstlisting}[style=MyBashStyle]
gcc -pthread funcoes.c thrd.c -o thrd
./thrd
\end{lstlisting}

%\textcolor{orange}{\faExclamationTriangle} Observação: a chave \textcolor{red}{\texttt{-lpthread}} indica que será usada a biblioteca \texttt{pthread} para Linux. Em algumas distribuições, você deve usar a chave \textcolor{red}{\texttt{-pthread}}.

\section{Passagem de parâmetros para \textit{threads}}
Vejamos agora como ocorre a passagem de parâmetros para os \textit{threads}. Antes, é preciso lembrar que na biblioteca PThreads, os \textit{threads} usam uma assinatura específica descrita na Figura~\ref{fig:assinaturaThread}. 

Para passar parâmetros para os \textit{threads}, é preciso encapsulá-los em uma estrutura e passar o endereço dessa estrutura para o \textit{thread}. Depois, já no escopo da função que implementa o \textit{thread}, esses parâmetros podem ser atribuídos às variáveis locais para serem utilizados.

\section{Retorno dos \textit{threads}}
De forma análoga à passagem de parâmetros, um \textit{thread} deve retornar um endereço de memória ou o endereço nulo (\texttt{NULL}).  

É importante destacar que, se um \textit{thread} for devolver um endereço de memória, essa posição de memória deve ter sido alocada dinamicamente. A razão para isso é que, após o término da função, todas as variáveis locais declaradas no escopo daquela função deixarão de existir. Portanto, um endereço alocado dinamicamente continuará existindo, mesmo após o término da função, até que a desalocação seja feita explicitamente (com a função \texttt{free}).

Também é preciso lembrar que esse endereço de memória devolvido ao final do \textit{thread}, precisa ser convertido para um tipo de dado específico. Caso essa operação não seja feita, o compilador poderá informar um erro de tipos na utilização da variável.

Como um exemplo de passagem de parâmetros, vejamos o exemplo a seguir. Trata-se de um programa \textit{multithread} que vai calcular a média dos números em um vetor. Nesse exemplo em particular, vamos usar um vetor de 100 posições e cada um dos quatro \textit{threads} calculará a soma parcial das suas respectivas partições (cada \textit{thread} calculará a soma de 25 elementos do vetor).

Note, logo no início da função que especifica o \textit{thread}, como os dados são extraídos do parâmetro \texttt{args} e atribuídos às variáveis locais. Essa estratégia facilita o uso posterior das variáveis na função.

Perceba também que a variável de retorno (\texttt{soma}) é um ponteiro. Esse ponteiro terá a memória alocada dinamicamente na função para que possa ser devolvida no final do \textit{thread}. 
\clearpage %<-- Adicionado por questões estéticas.

\section*{mediaThread.c}
\lstinputlisting[style=MyCStyle]{./Programas/Thread/mediaThreads.c}

\subsection{Exercício}
Compile e execute o programa anterior utilizando a seguinte linha de comando:

\begin{lstlisting}[style=MyBashStyle]
gcc -pthread auxFuncs.c mediaThreads.c -o mediaThreads
./mediaThreads
\end{lstlisting}

%\textcolor{orange}{\faExclamationTriangle} Observação: a chave \textcolor{red}{\texttt{-lpthread}} indica que será usada a biblioteca \texttt{pthread} para Linux. Em algumas distribuições, você deve usar a chave \textcolor{red}{\texttt{-pthread}}.



\section{Mais informações sobre programação \textit{multithread}}

No repositório \href{https://github.com/gradvohl/YAPTT}{\textit{Yet Another PThread Tutorial} -- YAPTT}, disponível no GitHub, há um tutorial (em inglês) bastante completo sobre programação usando PThreads.