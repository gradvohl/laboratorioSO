\section{Obtendo informações sobre processos}
Existem dois comandos para obtenção de informações sobre processos: \Comando{ps} e \Comando{top}.

O comando \Comando{ps} informa o status dos processos de forma sucinta. As informações que o comando \Comando{ps} apresenta são:

\begin{itemize}
\setlength{\itemsep}{1pt}\setlength{\parskip}{0pt}  \setlength{\parsep}{0pt}
\item \texttt{PID}: identificador do processo;
\item \texttt{TTY}: terminal onde o processo está sendo executado;
\item \texttt{TIME}: tempo de processamento;
\item \texttt{CMD}: comando instanciado.
\end{itemize}

O comando \Comando{top} é um pouco mais poderoso, pois reporta mais informações.  Entre tais informações estão:

\begin{itemize}
\setlength{\itemsep}{1pt}\setlength{\parskip}{0pt}  \setlength{\parsep}{0pt}
\item tempo em que o sistema está no ar;
\item carga média do sistema;
\item informações da CPU:
\begin{itemize}
\setlength{\itemsep}{1pt}\setlength{\parskip}{0pt}  \setlength{\parsep}{0pt}
\item porcentagem de tempo dedicada aos processos do usuário;
\item porcentagem de tempo dedicada aos processos do sistema;
\item porcentagem de tempo sem processamento (\textit{idle}).
\end{itemize}
\item informações sobre a memória:
\begin{itemize}
\setlength{\itemsep}{1pt}\setlength{\parskip}{0pt}  \setlength{\parsep}{0pt}
\item memória total;
\item memória  livre;
\item memória  compartilhada;
\end{itemize}
\item informações sobre os processos:
\begin{itemize}
\setlength{\itemsep}{1pt}\setlength{\parskip}{0pt}  \setlength{\parsep}{0pt}
\item \texttt{PID}: identificador do processo;
\item \texttt{USER}: nome do usuário;
\item \texttt{PRI}: prioridade;
\item \texttt{SIZE}: tamanho do processo em kbytes;
\item \texttt{RSS}: tamanho total de memória física do processo;
\item \texttt{SHARE}: tamanho total de memória compartilhada;
\item \texttt{STAT}:  estado do processo, que pode ser \texttt{S}  (\textit{sleeping}) ou \texttt{R} (\textit{running}).
\end{itemize}
\end{itemize}

\section{Exercício}
Utilize o comando \Comando{top} em sua máquina e tente identificar as informações providas pelo comando.

\chapter{Obtendo informações sobre os processos}
Neste capítulo, vamos verificar como obter informações sobre o próprio processo a partir dele mesmo.

\section{Obtendo informações sobre o processo, via programa}
É possível construir programas que interajam com o sistema operacional e obtenham algumas informações. Observe o código do programa a seguir:

\lstinputlisting[style=MyCStyle]{./Programas/Processos/infoProcesso.c}

\section{Exercício}
Compile o programa anterior e execute-o.

Antes de compilar o programa, mude para o diretório onde se encontra o arquivo  \texttt{infoProcesso.c}, com o seguinte comando:

\begin{lstlisting}[style=MyBashStyle]
cd Processos
\end{lstlisting}

Para compilar o programa utilize o comando a seguir:
\begin{lstlisting}[style=MyBashStyle]
gcc infoProcesso.c -o infoProcesso
\end{lstlisting}