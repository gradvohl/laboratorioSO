\chapter{Compartilhamento de memória}

Conforme discutido em sala de aula, é possível fazer com que dois ou mais processos compartilhem memória. Essa é uma forma para estabelecer a comunicação entre dois processos que estão no mesmo espaço de endereçamento (memória). Veremos como compartilhar a memória de dois processos a seguir.

\section{Primitivas para compartilhamento de memória}
As primitivas usadas para fazer o compartilhamento e acesso são:
\begin{itemize}
\setlength{\itemsep}{1pt}\setlength{\parskip}{0pt}  \setlength{\parsep}{0pt}
\item \texttt{shmget}: retorna o identificador do segmento de memória compartilhado;
\item \texttt{shmat}: anexa o segmento de memória compartilhado ao espaço de endereçamento do processo;
\item \texttt{shmdt}: desanexa o segmento de memória compartilhado ao espaço de endereçamento do processo.
\end{itemize}

 Detalharemos a seguir a sintaxe das primitivas. A primitiva \texttt{shmget} tem a seguinte sintaxe:

\begin{lstlisting}[style=MyCStyle, frame=none, numbers=none]
    int shmget(key_t key, size_t size, int shmflg);
\end{lstlisting}
 
onde os parâmetros são os seguintes: 
\begin{itemize}
    \item \texttt{key} contém o rótulo para o identificador do segmento de memória que será compartilhado. Esse rótulo (\texttt{key}) deve ser conhecido por todos os processos envolvidos no compartilhamento de memória.
    \item \texttt{size} é o tamanho do segmento a ser compartilhado, arredondado para o múltiplo do tamanho da página de memória\footnote{Veremos o conceito de \enquote{página de memória} quando abordarmos o assunto Memória Virtual.}.
    \item \texttt{shmflg} é uma constante indicando o tipo de operação que será realizada.  Geralmente será \texttt{IPC\_CREAT} para informar que o segmento será criado.
\end{itemize}
     
 O valor de retorno da primitiva \texttt{shmget} é o identificador do segmento de memória compartilhado indicado em \texttt{key}.

 Por sua vez, a primitiva \texttt{shmat} usa a seguinte sintaxe:

 \begin{lstlisting}[style=MyCStyle, frame=none, numbers=none]
    void *shmat(int shmid, const void *_Nullable shmaddr, int shmflg);
\end{lstlisting}
 
onde os parâmetros são os seguintes: 
\begin{itemize}
    \item \texttt{shmid} o identificador do segmento de memória compartilhado. Esse identificador deve ser obtido com a primitiva (\texttt{shmget}) anterior.
    \item \texttt{shmaddr} endereço ao qual o segmento de memória compartilhado será associado. Nesse caso, a recomendação é que se use \texttt{NULL} para o sistema operacional associar o segmento compartilhado ao endereço de memória mais adequado.
    \item \texttt{shmflg} é uma constante indicando como a associação de memória será realizada.  Recomenda-se que se coloque o valor zero.
\end{itemize}
     
 A primitiva \texttt{shmat} retorna o endereço do segmento de memória associado ou \texttt{-1} em caso de erro.


 Por sua vez, a primitiva \texttt{shmdt} usa a seguinte sintaxe:

 \begin{lstlisting}[style=MyCStyle, frame=none, numbers=none]
       int shmdt(const void *shmaddr);
\end{lstlisting}
 
O parâmetro que a primitiva recebe é o seguinte: 
\begin{itemize}
    \item \texttt{shmaddr} endereço ao qual o segmento de memória compartilhado está associado.
\end{itemize}
     
 A primitiva \texttt{shmdt} retorna zero, se tudo funcionou corretamente, ou \texttt{-1} em caso de erro.

\section{Exemplo de compartilhamento de memória entre processos distintos}
Observe o que os programas a seguir fazem. O primeiro é o programa \texttt{shm\_serv.c} que disponibilizará um segmento de memória que está no seu espaço de endereçamento. O segundo é o programa \texttt{shm\_cli.c} que acessará o segmento de memória compartilhado pelo primeiro processo.

%Adicionado por questões estéticas - Remova se necessário.
%\clearpage

\section*{\texttt{shm\_serv.c}}
\lstinputlisting[style=MyCStyle]{./Programas/CompartMem/shm_serv.c}


\section*{\texttt{shm\_cli.c}}
\lstinputlisting[style=MyCStyle]{./Programas/CompartMem/shm_cli.c}

\subsection{Exercício}

Primeiro, compile ambos os programas separadamente. Em seguida, abra duas janelas do terminal. Em uma das janelas execute o programa  \texttt{shm\_serv}. Na outra janela, execute o programa \texttt{shm\_cli}.

Antes de compilar o programa, mude para o diretório onde se encontram os arquivos \texttt{shm\_serv.c} e \texttt{shm\_cli.c} com o seguinte comando:
\begin{lstlisting}[style=MyBashStyle]
cd ../CompartMem
\end{lstlisting}


Para compilar o programa utilize as linhas de comando a seguir:
\begin{lstlisting}[style=MyBashStyle]
gcc shm_serv.c -o shm_serv
gcc shm_cli.c -o shm_cli
\end{lstlisting}

Agora, em uma das janelas execute primeiro o programa \texttt{./shm\_serv}. Note que o programa ficará em execução, esperando a leitura dos dados pelo segundo programa.

Depois, na segunda janela, execute o programa \texttt{./shm\_cli}. Veja o resultado.

Reflita e responda às perguntas a seguir: 
\begin{enumerate}
    \item Que tipo de mecanismo de sincronização foi usado para fazer com que o programa \texttt{./shm\_serv} esperasse até que o programa \texttt{./shm\_cli} lesse os dados?
    \item Em que linhas fica evidente o mecanismo de sincronização utilizado?
\end{enumerate}
    
