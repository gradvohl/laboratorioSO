\chapter{Compartilhamento de memória}

Conforme discutido em sala de aula, é possível fazer com que dois ou mais processos compartilhem memória. Essa é uma forma para fazer com que dois processos possam se comunicar.

\section{Primitivas para compartilhamento de memória}
As primitivas usadas para fazer o compartilhamento e acesso são:
\begin{itemize}
\setlength{\itemsep}{1pt}\setlength{\parskip}{0pt}  \setlength{\parsep}{0pt}
\item \texttt{shmget}: retorna o identificador do segmento de memória compartilhado;
\item \texttt{shmat}: anexa o segmento de memória compartilhado ao espaço de endereçamento do processo;
\item \texttt{shmdt}: desanexa o segmento de memória compartilhado ao espaço de endereçamento do processo.
\end{itemize}

Observe o que os programas a seguir fazem. O primeiro é o programa \texttt{shm\_serv.c} que disponibiliza um segmento de memória. O segundo é o programa \texttt{shm\_cli.c} que acessa o segmento compartilhado.

\section*{\texttt{shm\_serv.c}}
\lstinputlisting[style=MyCStyle]{./Programas/CompartMem/shm_serv.c}

\newpage
\section*{\texttt{shm\_cli.c}}
\lstinputlisting[style=MyCStyle]{./Programas/CompartMem/shm_cli.c}

\section{Exercício}

Compile ambos os programas e, em seguida, execute em uma janela o programa  \texttt{shm\_serv} e em outra janela o programa \texttt{shm\_cli}.

Antes de compilar o programa, mude para o diretório onde se encontram os arquivos \texttt{shm\_serv.c} e \texttt{shm\_cli.c} , com o seguinte comando:
\begin{lstlisting}[style=MyBashStyle]
cd ../CompartMem
\end{lstlisting}


Para compilar o programa utilize as linhas de comando a seguir:
\begin{lstlisting}[style=MyBashStyle]
gcc shm_serv.c -o shm_serv
gcc shm_cli.c -o shm_cli
\end{lstlisting}

Agora, em uma das janela execute primeiro o programa \texttt{./shm\_serv.o} e, depois, na segunda janela execute o programa \texttt{./shm\_cli.o}. Veja o resultado.