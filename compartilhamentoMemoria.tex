\chapter{Compartilhamento de memória}

Conforme discutido em sala de aula, é possível fazer com que dois ou mais processos compartilhem memória. Essa é uma forma para estabelecer a comunicação entre dois processos que estão no mesmo espaço de endereçamento (memória). Veremos como compartilhar a memória de dois processos a seguir.

\section{Primitivas para compartilhamento de memória}
As primitivas usadas para fazer o compartilhamento e acesso são:
\begin{itemize}
\setlength{\itemsep}{1pt}\setlength{\parskip}{0pt}  \setlength{\parsep}{0pt}
\item \texttt{shmget}: retorna o identificador do segmento de memória compartilhado;
\item \texttt{shmat}: anexa o segmento de memória compartilhado ao espaço de endereçamento do processo;
\item \texttt{shmdt}: desanexa o segmento de memória compartilhado ao espaço de endereçamento do processo.
\end{itemize}

Observe o que os programas a seguir fazem. O primeiro é o programa \texttt{shm\_serv.c} que disponibilizará um segmento de memória que está no seu espaço de endereçamento. O segundo é o programa \texttt{shm\_cli.c} que acessará o segmento de memória compartilhado pelo primeiro processo.

%Adicionado por questões estéticas - Remova se necessário.
\clearpage

\section*{\texttt{shm\_serv.c}}
\lstinputlisting[style=MyCStyle]{./Programas/CompartMem/shm_serv.c}


\section*{\texttt{shm\_cli.c}}
\lstinputlisting[style=MyCStyle]{./Programas/CompartMem/shm_cli.c}

\subsection{Exercício}

Primeiro, compile ambos os programas separadamente. Em seguida, abra duas janelas do terminal. Em uma das janelas execute o programa  \texttt{shm\_serv}. Na outra janela, execute o programa \texttt{shm\_cli}.

Antes de compilar o programa, mude para o diretório onde se encontram os arquivos \texttt{shm\_serv.c} e \texttt{shm\_cli.c} com o seguinte comando:
\begin{lstlisting}[style=MyBashStyle]
cd ../CompartMem
\end{lstlisting}


Para compilar o programa utilize as linhas de comando a seguir:
\begin{lstlisting}[style=MyBashStyle]
gcc shm_serv.c -o shm_serv
gcc shm_cli.c -o shm_cli
\end{lstlisting}

Agora, em uma das janelas execute primeiro o programa \texttt{./shm\_serv}. Note que o programa ficará em execução, esperando a leitura dos dados pelo segundo programa.

Depois, na segunda janela, execute o programa \texttt{./shm\_cli}. Veja o resultado.

Reflita e responda às perguntas a seguir: 
\begin{enumerate}
    \item Que tipo de mecanismo de sincronização foi usado para fazer com que o programa \texttt{./shm\_serv} esperasse até que o programa \texttt{./shm\_cli} lesse os dados?
    \item Em que linhas fica evidente o mecanismo de sincronização utilizado?
\end{enumerate}
    
