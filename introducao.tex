\chapter{Introdução}
O objetivo deste texto é descrever os exercícios usados no laboratório da disciplina Sistemas Operacionais. Essa disciplina é oferecida na Faculdade de Tecnologia da Universidade Estadual de Campinas (FT/UNICAMP) para os cursos Bacharelado em Sistemas de Informação e Tecnologia em Análise e Desenvolvimento de Sistemas.

Esse material pode ser utilizado por qualquer pessoa, de qualquer curso ou instituição, desde que respeitadas as condições da licença CC-BY-4.0, descrita na página \pageref{chp:licenca}. Informações de como obter o material também estão nessa página.

Supõe-se que o sistema operacional utilizado será o Linux \faLinux. Portanto, todos os comandos descritos neste texto são para o Linux. Recomenda-se que o leitor navegue sequencialmente pelo texto. Assim, terá melhor aproveitamento do laboratório.

Alguns comandos básicos para o sistema Linux estão na \Cref{tab:comandosLinux} a seguir:

\begin{table}[!htb]
\begin{center}
    \caption{Lista de comandos comuns no Linux.}\label{tab:comandosLinux}
\begin{tabular}{@{}lp{13cm}@{}}
\toprule
\textbf{Comando}       & \textbf{Significado} \\ \midrule
\ComandoParametros{cd}{dir}        & Muda para o diretório \textcolor{red}{\texttt{dir}}.       \\
\multirow{2}{*}{\ComandoParametros{gedit}{arquivo}\,\&} & Abre o \textcolor{red}{\texttt{arquivo}} no editor de textos em segundo 
plano e libera o terminal para outros comandos. Note o \& no fim da linha. \\
\Comando{ls} & Lista os arquivos locais.        \\
\ComandoParametros{unzip}{arq.zip} & Descompacta o arquivo \textcolor{red}{\texttt{arq.zip}}.   \\ \bottomrule
\end{tabular}
\end{center}
\end{table}

Todos os comandos que serão utilizados nesse tutorial serão executados no interpretador da linha de comandos (\textit{shell}), também chamado de terminal.

Há ainda algumas dicas de teclas para os usuários iniciantes no \textit{bash} (o interpretador de comandos padrão no Linux). Elas estão resumidas na \Cref{tab:teclasBash} a seguir.

\begin{table}[!htb]
\begin{center}
    \caption{Teclas úteis no \textit{bash}.}\label{tab:teclasBash}
\begin{tabular}{@{}cl@{}}
\toprule
\textbf{Teclas}     & \textbf{Significado} \\ \midrule
\keys{\arrowkeyup} & Repete o último comando.        \\
\keys{\arrowkeydown} & Repete o próximo comando.        \\
\keys{\ctrl + c} & Envia um sinal de término para o processo.\\
\keys{\tab} & Completa o nome do comando ou do arquivo.\\ 
\keys{\esc + d} & Apaga a próxima palavra a frente do cursor.\\
\keys{\ctrl + k} & Apaga do cursor até o final da linha.\\
\keys{\ctrl + a} ou \keys{Home} & Navega para o início da linha.\\
\keys{\ctrl + e} ou \keys{End} & Navega para o final da linha.\\
\keys{\ctrl + \arrowkeyleft} & Navega para a palavra anterior o cursor.\\
\keys{\ctrl + \arrowkeyright} & Navega para a próxima palavra a frente do cursor.\\
\bottomrule
\end{tabular}
\end{center}
\end{table}

\section{Manual dos comandos Linux}
Se a distribuição Linux que você está utilizando for instalada por completo, então o manual de todos os comandos e funções deve estar disponível no próprio a partir da linha de comando. Por exemplo, se você quiser saber para o que serve o comando \Comando{ls}, basta utilizar o comando a seguir. %Note que não é necessário você digitar o símbolo \$. Esse símbolo, no início da linha, indica que o terminal está apto a receber os comandos. Isso vale para todos os comandos desse ponto em diante.

\begin{lstlisting}[style=MyBashStyle]
man ls
\end{lstlisting}

Caso queira saber informações específicas sobre uma determinada função que utilizará em um código-fonte (geralmente na linguagem C), você também pode utilizar o comando \Comando{man}. Veja o exemplo a seguir para obter informações sobre a função \texttt{mallloc}.

\begin{lstlisting}[style=MyBashStyle]
man malloc
\end{lstlisting}
