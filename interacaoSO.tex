\chapter{Interação com o Sistema Operacional}
Neste capítulo, veremos alguns comandos básicos para a interação com o sistema operacional Linux. 

\section{Interagindo com o sistema operacional}
O comando básico para obter informações sobre o sistema operacional é o
%\begin{center}  \texttt{uname -a} \end{center}

\begin{lstlisting}[style=MyBashStyle]
uname -a
\end{lstlisting}

Observe a saída desse comando:
\begin{lstlisting}[style=outputStyle]
Linux grid1.cna.unicamp.br 2.4.20-8 #1 Thu Mar 13 17:18:24 EST 2003 i686 athlon i386 GNU/Linux
\end{lstlisting}

Entre as informações presentes na saída desse comando estão:
\begin{itemize}
\setlength{\itemsep}{1pt}\setlength{\parskip}{0pt}  \setlength{\parsep}{0pt}
\item o nome do sistema operacional;
\item o nome da máquina;
\item versão do kernel;
\item plataforma de hardware.
\end{itemize}

\subsection{Exercício}
Utilize o comando \ComandoParametros{uname}{-a} em sua máquina e tente identificar a saída do comando.

\section{Obtendo informações sobre o hardware}
Para obter informações sobre o hardware no qual o sistema operacional está funcionando, utilizaremos dois comandos. 

\subsection{Informações sobre a unidade central de processamento}
Para buscar informações sobre a unidade central de processamento (CPU), utilizaremos o comando  \Comando{lscpu}. Esse comando buscará informações armazenadas no arquivo \texttt{/proc/cpuinfo} e em outros locais específicos do sistema operacional.

\subsection{Exercício}
Utilize o comando \Comando{lscpu} em sua máquina e tente identificar quais as informações esse comando fornece.

Depois, use o comando a seguir e verifique que informações esse comando fornece. Compare a saída do comando \Comando{lscpu} com a saída desse comando.

\begin{lstlisting}[style=MyBashStyle]
cat /proc/cpuinfo
\end{lstlisting}


\subsection{Informações sobre a memória}
Para obter informações sobre a memória, podemos utilizar o comando \Comando{lsmem}. Esse comando trará, resumidamente, as informações sobre a memória. 

Dentre essas informações que o comando \Comando{lsmem} provê estão a faixa de endereços de memória (\texttt{RANGE}), o tamanho de cada faixa (\texttt{SIZE}), o estado (\texttt{STATE}), se é um bloco removível, e os blocos correspondentes (\texttt{BLOCK}). Ao final, o comando indica o tamanho do bloco de memória e a quantidade total de memória \textit{online} e \textit{offline} (desativada).

Para informações mais completas sobre a memória do computador que você está utilizando, abra o arquivo \texttt{/proc/meminfo}.

\subsection{Exercício}
Utilize o comando \Comando{lsmem} em sua máquina e tente identificar quais as informações esse comando fornece.

Depois, use o comando \Comando{cat} \texttt{/proc/meminfo} e verifique que informações esse comando fornece. Compare a saída do comando \Comando{lsmem} com a saída desse comando.
