\chapter{Interação com o Sistema Operacional}
\section{Interagindo com o sistema operacional}
O comando básico para obter informações sobre o sistema operacional é o
%\begin{center}  \texttt{uname -a} \end{center}

\begin{lstlisting}[style=MyBashStyle]
uname -a
\end{lstlisting}



Observe a saída desse comando:
\begin{lstlisting}[style=outputStyle]
Linux grid1.cna.unicamp.br 2.4.20-8 #1 Thu Mar 13 17:18:24 EST 2003 i686 athlon i386 GNU/Linux
\end{lstlisting}

Entre as informações presentes na saída desse comando estão:
\begin{itemize}
\setlength{\itemsep}{1pt}\setlength{\parskip}{0pt}  \setlength{\parsep}{0pt}
\item o nome do sistema operacional;
\item o nome da máquina;
\item versão do kernel;
\item plataforma de hardware.
\end{itemize}

\section{Exercício}
Utilize o comando \ComandoParametros{uname}{-a} em sua máquina e tente identificar a saída do comando.
