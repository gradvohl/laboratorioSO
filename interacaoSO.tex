\chapter{Interação com o Sistema Operacional}
Neste capítulo, veremos alguns comandos básicos para a interação com o sistema operacional Linux. 

\section{Interagindo com o sistema operacional}
O comando básico para obter informações sobre o sistema operacional está a seguir.
%\begin{center}  \texttt{uname -a} \end{center}

\begin{lstlisting}[style=MyBashStyle]
uname -a
\end{lstlisting}

Observe a saída desse comando:
\begin{lstlisting}[style=outputStyle]
Linux grid1.cna.unicamp.br 2.4.20-8 #1 Thu Mar 13 17:18:24 EST 2003 i686 athlon i386 GNU/Linux
\end{lstlisting}

Entre as informações presentes na saída desse comando estão:
\begin{itemize}
\setlength{\itemsep}{1pt}\setlength{\parskip}{0pt}  \setlength{\parsep}{0pt}
\item o nome do sistema operacional;
\item o nome da máquina;
\item versão do kernel;
\item plataforma de hardware.
\end{itemize}

\subsection{Exercício}
Utilize o comando \ComandoParametros{uname}{-a} em sua máquina e tente identificar a saída do comando.


\subsection{Mais informações sobre a distribuição do sistema operacional}
Outra possibilidade para obter mais informações sobre a distribuição do sistema operacional utilizado é consultar o arquivo \texttt{/etc/os-release}. Para isso use o comando a seguir.

\begin{lstlisting}[style=MyBashStyle]
cat /etc/os-release 
\end{lstlisting}

Observe a saída que o comando anterior produz.
\begin{lstlisting}[style=outputStyle]
NAME="Rocky Linux"
VERSION="9.5 (Blue Onyx)"
ID="rocky"
ID_LIKE="rhel centos fedora"
VERSION_ID="9.5"
PLATFORM_ID="platform:el9"
PRETTY_NAME="Rocky Linux 9.5 (Blue Onyx)"
ANSI_COLOR="0;32"
LOGO="fedora-logo-icon"
CPE_NAME="cpe:/o:rocky:rocky:9::baseos"
HOME_URL="https://rockylinux.org/"
VENDOR_NAME="RESF"
VENDOR_URL="https://resf.org/"
BUG_REPORT_URL="https://bugs.rockylinux.org/"
SUPPORT_END="2032-05-31"
ROCKY_SUPPORT_PRODUCT="Rocky-Linux-9"
ROCKY_SUPPORT_PRODUCT_VERSION="9.5"
REDHAT_SUPPORT_PRODUCT="Rocky Linux"
REDHAT_SUPPORT_PRODUCT_VERSION="9.5"
\end{lstlisting}

Dentre as informações apresentadas estão o nome da distrição Linux (nesse caso, \enquote{\texttt{Rocky Linux}}), a versão (\enquote{\texttt{9.5 (Blue Onyx)}}) e outros detalhes específicos da versão do sistema operacional utilizado.

\section{Obtendo informações sobre o hardware}
Para obter informações sobre o hardware no qual o sistema operacional está funcionando, utilizaremos dois comandos. 

\subsection{Informações sobre a unidade central de processamento}
Para buscar informações sobre a unidade central de processamento (CPU), utilizaremos o comando  \Comando{lscpu}. Esse comando buscará informações armazenadas no arquivo \texttt{/proc/cpuinfo} e em outros locais específicos do sistema operacional.

\subsection{Exercício}
Utilize o comando \Comando{lscpu} em sua máquina e tente identificar quais as informações esse comando fornece.

Depois, use o comando a seguir e verifique que informações esse comando fornece. Compare a saída do comando \Comando{lscpu} com a saída desse comando.

\begin{lstlisting}[style=MyBashStyle]
cat /proc/cpuinfo
\end{lstlisting}


\subsection{Informações sobre a memória}
Para obter informações sobre a memória, podemos utilizar o comando \Comando{lsmem}. Esse comando trará, resumidamente, as informações sobre a memória. 

Dentre essas informações que o comando \Comando{lsmem} provê estão a faixa de endereços de memória (\texttt{RANGE}), o tamanho de cada faixa (\texttt{SIZE}), o estado (\texttt{STATE}), se é um bloco removível, e os blocos correspondentes (\texttt{BLOCK}). Ao final, o comando indica o tamanho do bloco de memória e a quantidade total de memória \textit{online} e \textit{offline} (desativada).

Para informações mais completas sobre a memória do computador que você está utilizando, abra o arquivo \texttt{/proc/meminfo}.

\subsection{Exercício}
Utilize o comando \Comando{lsmem} em sua máquina e tente identificar quais as informações esse comando fornece.

Depois, use o comando \semaspas{\Comando{cat} \texttt{/proc/meminfo}} e verifique que informações esse comando fornece. Compare a saída do comando \Comando{lsmem} com a saída desse comando.


\section{Informações sobre o sistema por meio de um código-fonte}
Todas as informações sobre o hardware, isto é, informações sobre CPU e memória podem ser obtidas por meio de um código-fonte. Para isso, observe o código fonte a seguir.

\clearpage
\lstinputlisting[style=MyCStyle]{./Programas/InfoSistema/infoSistema.c}

\subsection{Exercício}
Para ver o resultado do programa anterior, compile-o e execute-o. Antes de compilar o programa, mude para o diretório onde se encontra o arquivo  \texttt{infoSistema.c}, com o seguinte comando:

\begin{lstlisting}[style=MyBashStyle]
cd InfoSistema
\end{lstlisting}

Fique atento, pois o interpretador da linha de comando (o \textit{shell}) diferencia as letras maiúsculas das letras minúsculas.

Para compilar o programa, utilize o comando a seguir.
\begin{lstlisting}[style=MyBashStyle]
gcc infoSistema.c -o infoSistema
\end{lstlisting}

Para executar o programa, execute a linha de comando a seguir.
\begin{lstlisting}[style=MyBashStyle]
./infoSistema
\end{lstlisting}